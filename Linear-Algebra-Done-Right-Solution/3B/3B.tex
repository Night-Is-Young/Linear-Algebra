\documentclass{ctexart}
\usepackage{geometry}
\usepackage[dvipsnames,svgnames]{xcolor}
\usepackage[strict]{changepage}
\usepackage{framed}
\usepackage{enumerate}
\usepackage{amsmath,amsthm,amssymb}
\usepackage{enumitem}
\usepackage{solution}

\allowdisplaybreaks
\geometry{left=2cm, right=2cm, top=2.5cm, bottom=2.5cm}

\begin{document}\pagestyle{empty}
\begin{problem}[1.]
    给出一例线性映射$T$满足$\dim\nul T=3$且$\dim\range T=2$.
\end{problem}
\begin{solution}[Solution.]
    令$T:\R^5\to\R^2$满足
    $$T(x_1,\cdots,x_5)=(x_1,x_2)$$
    于是$\nul T=\left\{(0,0,x,y,z)\in\R^5:x,y,z\in\R\right\}$,故$\dim\nul T=3$.\\
    又$\range T=\R^2$,于是$\dim\range T=2$.这样就构造出了符合题设的线性映射$T$.
\end{solution}
\begin{problem}[2.]
    设$S,T\in\mathcal{L}(V)$使得$\range S\subseteq\null T$,证明$(ST)^2=\mbf{0}$
\end{problem}
\begin{proof}
    对于任意$v\in V$都有$S(Tv)\in\range S$,于是$STv\in\null T$,于是$T(STv)=\mbf{0}$,
    于是$S(T(STv))=S(\mbf{0})=\mbf{0}$,进而$(ST)^2v=\mbf{0}$.
    从而$(ST)^2=\mbf{0}$.
\end{proof}
\begin{problem}[3.]
    设$v_1,\cdots,v_m$是$V$中一组向量,定义$T\in\mathcal{L}(\F^m,V)$为
    $$T(z_1,\cdots,z_m)=z_1v_1+\cdots+z_mv_m$$
    \begin{enumerate}[label=\tbf{(\arabic*)}]
        \item 当$v_1,\cdots,v_m$张成$V$时,指出$T$具有的性质.
        \item 当$v_1,\cdots,v_m$在$V$中线性无关时,指出$T$具有的性质.
    \end{enumerate}
\end{problem}
\begin{solution}[Solution.]
    \begin{enumerate}[label=\tbf{(\arabic*)}]
        \item $T$是满射.
        \item $T$是单射.
    \end{enumerate}
\end{solution}
\begin{problem}[4.]
    证明:集合$\left\{T\in\mathcal{L}\left(\R^5,\R^4\right):\dim\nul T>2\right\}$不是$\mathcal{L}\left(\R^5,\R^4\right)$的子空间.
\end{problem}
\begin{proof}
    取$T_1,T_2\in\mathcal{L}\left(\R^5,\R^4\right)$满足
    $$T_1(x_1,\cdots,x_5)=(x_1,x_2,0,0),T_2(x_1,\cdots,x_5)=(0,0,x_3,x_4)$$
    易知$\nul T_1=\left\{(0,0,x,y,z)\in\R^5:x,y,z\in\R\right\},\nul T_2=\left\{(x,y,0,0,z)\in\R^5:x,y,z\in\R\right\}$.\\
    于是$\dim\nul T_1=\dim\nul T_2=3>2$,故$T_1,T_2\in\left\{T\in\mathcal{L}\left(\R^5,\R^4\right):\dim\nul T>2\right\}$.\\
    然而$(T_1+T_2)(x_1,\cdots,x_5)=(x_1,x_2,x_3,x_4)$,$\nul(T_1+T_2)=\left\{(0,0,0,0,x)\in\F^5:x\in\F\right\}$,于是$\dim\nul (T_1+T_2)=1$,
    进而$T_1+T_2\notin\left\{T\in\mathcal{L}\left(\R^5,\R^4\right):\dim\nul T>2\right\}$,于是该集合对加法不封闭,不是$\mathcal{L}\left(\R^5,\R^4\right)$的子空间.
\end{proof}
\begin{problem}[5.]
    给出一例使得$\range T=\nul T$的$T\in\mathcal{L}(\R^4)$.
\end{problem}
\begin{solution}[Solution.]
    对于任意$(x_1,x_2,x_3,x_4)\in\R^4$,令$T(x_1,x_2,x_3,x_4)=(x_1,x_2,0,0)$即可满足题意.
\end{solution}
\begin{problem}[6.]
    试证明不存在$T\in\mathcal{L}(\R^5)$使得$\range T=\nul T$.
\end{problem}
\begin{proof}
    由线性映射基本定理可知$\dim\range T+\dim\nul T=\dim V=5$.\\
    于是不存在$T$满足$\dim\range T=\dim\nul T$,因而$\range T\neq\nul T$对所有$T\in\mathcal{L}(\R^5)$成立.
\end{proof}
\begin{problem}[7.]
    设$V$和$W$是有限维的且满足$2\leqslant\dim V\leqslant\dim W$,试证明$\left\{T\in\mathcal{L}(V,W):T\text{不是单射}\right\}$不是$\mathcal{L}(V,W)$的子空间.
\end{problem}
\begin{proof}
    设$v_1,\cdots,v_n$为$V$的一个基,$w_1,\cdots,w_n$是$W$中的一组线性无关向量.\\
    令$T_1,T_2\in\mathcal{L}(V,W)$满足
    $$\left\{\begin{array}{l}
        T_1v_1=\mbf{0}\\
        T_1v_k=w_k,k\in\left\{2,\cdots,n\right\}
    \end{array}\right.
    \left\{\begin{array}{l}
        T_2v_2=\mbf{0}\\
        T_2v_k=w_k,k\in\left\{1,\cdots,n\right\}\backslash\left\{2\right\}
    \end{array}\right.$$
    不难证明$T_1,T_2$均不是单射.然而,对于任意一组标量$a_1,\cdots,a_n\in\F$有
    $$\begin{aligned}
        \left(T_1+T_2\right)\left(a_1v_1+\cdots+a_nv_n\right)
        &= a_1T_1v_1+\cdots+a_nT_1v_n+a_1T_2v_1+\cdots+a_nT_2v_n \\
        &= a_2w_2+\cdots+a_nw_n+a_1w_1+\cdots+a_nw_n \\
        &= a_1w_1+a_2w_2+\cdots+2a_nw_n \\
    \end{aligned}$$
    因为$w_1,\cdots,w_n$线性无关,于是$T_1+T_2$是单射,于是该集合对加法不封闭.命题得证.
\end{proof}
\begin{problem}[8.]
    设$V$和$W$是有限维的且满足$2\leqslant\dim W\leqslant\dim V$,试证明$\left\{T\in\mathcal{L}(V,W):T\text{不是满射}\right\}$不是$\mathcal{L}(V,W)$的子空间.
\end{problem}
\begin{proof}
    设$v_1,\cdots,v_n$为$V$的一个基,$w_1,\cdots,w_m$是$W$的一个基.\\
    令$T_1,T_2\in\mathcal{L}(V,W)$满足
    $$\left\{\begin{array}{l}
        T_1v_1=\mbf{0}\\
        T_1v_j=w_j,j\in\left\{2,\cdots,m\right\}\\
        T_1v_k=\mbf{0},k\in{m+1,\cdots,n}
    \end{array}\right.
    \left\{\begin{array}{l}
        T_2v_2=\mbf{0}\\
        T_2v_j=w_j,j\in\left\{1,\cdots,m\right\}\backslash\left\{2\right\} \\
        T_2v_k=\mbf{0},k\in{m+1,\cdots,n}
    \end{array}\right.$$
    不难证明$T_1,T_2$均不是满射.然而,对于任意一组标量$a_1,\cdots,a_n\in\F$有
    $$\begin{aligned}
        \left(T_1+T_2\right)\left(a_1v_1+\cdots+a_nv_n\right)
        &= a_1T_1v_1+\cdots+a_nT_1v_n+a_1T_2v_1+\cdots+a_nT_2v_n \\
        &= a_1w_1+a_2w_2+\cdots+2a_mw_m
    \end{aligned}$$
    由于$w_1,\cdots,w_m$是$W$的基,于是$T_1+T_2$是满射,于是该集合对加法不封闭,命题得证.
\end{proof}
\begin{problem}[9.]
    设$T\in\mathcal{L}(V,W)$是单射,$v_1,\cdots,v_n$在$V$中线性无关.试证明$Tv_1,\cdots,Tv_n$在$W$中线性无关.
\end{problem}
\begin{proof}
    设一组标量$a_1,\cdots,a_n\in\F$,于是$T\left(a_1v_1+\cdots+a_nv_n\right)=a_1Tv_1+\cdots+a_nTv_n$.\\
    由$v_1,\cdots,v_n$线性无关可得当且仅当$a_1=\cdots=a_n=0$时$a_1v_1+\cdots+a_nv_n=\mbf{0}$.\\
    又$T(\mbf{0})=\mbf{0}$,于是$a_1Tv_1+\cdots+a_nTv_n=\mbf{0}$当且仅当$a_1=\cdots=a_n=0$.\\
    于是$Tv_1,\cdots,Tv_n$在$W$中线性无关.
\end{proof}
\begin{problem}[10.]
    设$V=\span(v_1,\cdots,v_n)$,$T\in\mathcal{L}(V,W)$.试证明$\range T=\span(Tv_1,\cdots,Tv_n)$.
\end{problem}
\begin{proof}
    由于$V=\span(v_1,\cdots,v_n)$,故对于任意$v\in V$均存在一组标量$a_1,\cdots,a_n\in\F$使得
    $$v=a_1v_1+\cdots+a_nv_n$$
    于是对于任意$Tv\in\range T$都有
    $$\begin{aligned}
        Tv 
        &= T\left(a_1v_1+\cdots+a_nv_n\right) \\
        &= a_1Tv_1+\cdots+a_nTv_n
    \end{aligned}$$
    从而$\range T=\span\left(Tv_1,\cdots Tv_n\right)$,命题得证.
\end{proof}
\begin{problem}[11.]
    设$V$是有限维的,$T\in\mathcal{L}(V,W)$.\\
    试证明:存在$V$的一个子空间$U$使得$U\cap\nul T=\left\{\mbf{0}\right\}$且$\range T=\left\{Tu:u\in U\right\}$.
\end{problem}
\begin{proof}
    设$u_1,\cdots,v_m$为$\nul T$的一个基,于是它可以被扩展为$V$的一个基$u_1,\cdots,u_n,v_1,\cdots,v_m$.\\
    根据线性映射基本定理可知$Tv_1,\cdots,Tv_m$为$\range T$的基.\\
    于是令$U=\text{span}(v_1,\cdots,v_m)$即可满足题意.
\end{proof}
\begin{problem}[12.]
    设线性映射$T:\F^4\to\F^2$使得
    $$\nul T=\left\{(x_1,x_2,x_3,x_4)\in\F^4:x_1=5x_2,x_3=7x_4\right\}$$
    试证明$T$是满射.
\end{problem}
\begin{proof}
    不难发现$(5,1,0,0),(0,0,7,1)$是$\nul T$的一个基,于是$\dim\nul T=2$.\\
    由线性映射基本定理,$\dim\range T=\dim V-\dim\nul T=2=\dim\R^2$,从而$\dim\range T=\R^2$,故$T$是满射.
\end{proof}
\begin{problem}[13.]
    设$U$是$\R^8$的子空间,$\dim U=3$,线性映射$T:\R^8\to\R^5$且$\nul T=U$.试证明$T$是满射.
\end{problem}
\begin{proof}
    由线性映射基本定理,$\dim\range T=\dim V-\dim\nul T=\dim V-\dim U=8-3=5$,于是$\range T=\R^5$,故$T$是满射.
\end{proof}
\begin{problem}[14.]
    试证明:不存在$T\in\mathcal{L}(\F^5,\F^2)$使得
    $$\nul T=\left\{(x_1,x_2,x_3,x_4,x_5)\in\F^5:x_1=3x_2\text{且}x_3=x_4=x_5\right\}$$
\end{problem}
\begin{proof}
    注意到$(3,1,0,0,0),(0,0,1,1,1)$是$\nul T$的一个基,于是$\dim\nul T=2$.\\
    根据线性映射基本定理可知$\dim\range T=\dim V-\dim\nul 5-2=3>2$,于是不存在这样的线性映射.
\end{proof}
\begin{problem}[15.]
    设$V$上的线性映射$T$,满足$T$的零空间和值域都是有限维的.试证明:$V$是有限维的.
\end{problem}
\begin{proof}
    不妨设$\dim\nul T=n,\dim\range T=m$,于是$m,n\in\N$.\\
    于是$\dim V=m+n\in\N$,故$V$是有限维的,命题得证.
\end{proof}
\begin{problem}[16.]
    设$V$和$W$都是有限维的,试证明:存在$V$到$W$的单射,当且仅当$\dim V\leqslant\dim U$.
\end{problem}
\begin{proof}
    首先证明$\dim V\leqslant\dim W$时存在这样的单射.\\
    设$v_1,\cdots,v_n$是$V$的一个基,$w_1,\cdots,w_n$是$W$中的一组线性无关的向量.\\
    设$T\in\mathcal{L}(V,W)$使得$Tv_k=w_k,k\in\left\{1,\cdots,n\right\}$.\\
    对于$v\in V$,设一组标量$a_1,\cdots,a_n$满足$v=a_1v_1+\cdots+a_nv_n$.于是
    $$\begin{aligned}
        Tv
        &= T\left(a_1v_1+\cdots+a_nv_n\right) \\
        &= a_1Tv_1+\cdots+a_nTv_n \\
        &= a_1w_1+\cdots+a_nw_n
    \end{aligned}$$
    又$w_1,\cdots,w_n$线性无关,于是$Tv=\mbf{0}$当且仅当$a_1=\cdots=a_n=0$,此时$v=\mbf{0}$,
    即$\nul T=\left\{\mbf{0}\right\}$,于是$T$为单射.\\
    我们已经证明,当$\dim V>\dim W$时,不存在$V$到$W$的单射.\\
    综上,命题得证.
\end{proof}
\begin{problem}[17.]
    设$V$和$W$都是有限维的,试证明:存在$V$到$W$的满射,当且仅当$\dim V\geqslant\dim U$.
\end{problem}
\begin{proof}
    首先证明$\dim V\geqslant\dim W$时存在这样的满射.\\
    设$w_1,\cdots,w_m$是$W$的一个基,$v_1,\cdots,v_n$是$V$的一个基.\\
    设$T\in\mathcal{L}(V,W)$使得
    $$\left\{\begin{array}{l}
        Tv_k=w_k,k\in{1,\cdots,m}\\
        Tv_k=\mbf{0},k\in{m+1,\cdots,n}
    \end{array}\right.$$
    对于任意$w\in W$,存在一组标量$a_1,\cdots,a_m$使得$w=a_1w_1+\cdots+a_mw_m$,于是
    $$\begin{aligned}
        w
        &= a_1w_1+\cdots+a_mw_m \\
        &= a_1Tv_1+\cdots+a_mTv_m \\
        &= T\left(a_1v_1+\cdots+a_mv_m\right)
    \end{aligned}$$
    又$v_1,\cdots,v_m$线性无关,于是$a_1v_1+\cdots+a_mv_m$唯一对应了$V$中的一个向量.
    于是$T$为满射.\\
    我们已经证明,当$\dim V<\dim W$时,不存在$V$到$W$的满射.\\
    综上,命题得证.
\end{proof}
\begin{problem}[18.]
    设$V$和$W$都是有限维的,$U$是$V$的子空间.试证明:存在$T\in\mathcal{L}(V,W)$使得$\nul T=U$,当且仅当$\dim U\geqslant\dim V-\dim W$.
\end{problem}
\begin{proof}
    设$u_1,\cdots,u_m$为$U$的一组基,于是它可以被扩展为$V$的一组基$u_1,\cdots,u_m,v_1,\cdots,v_n$.\\
    于是$\dim U=m,\dim V=m+n$.\\
    一方面,当$\dim U\geqslant\dim V-\dim W$时有$\dim W\geqslant n$,于是存在$w_1,\cdots,w_n$在$W$中线性无关.\\
    令$T\in\mathcal{L}(V,W)$满足
    $$\left\{\begin{array}{l}
        Tu_j=\mbf{0},j\in\left\{1,\cdots,m\right\}\\
        Tv_k=w_k,k\in\left\{1,\cdots,n\right\}
    \end{array}\right.$$
    于是$\nul T=\text{span}\left(u_1,\cdots,u_m\right)=U$,于是存在这样的$T$满足题意.\\
    另一方面,当存在$T\in\mathcal{L}(V,W)$使得$\nul T=U$时,应对于任意$v\in V\backslash U$有$Tv\neq\mbf{0}$.\\
    即对于任意$a_1,\cdots,a_m\in\F$和不全为$0$的$b_1,\cdots,b_n\in\F$,都有
    $$\begin{aligned}
        Tv
        &= T\left(a_1u_1+\cdots+a_mu_m+b_1v_1+\cdots+b_nv_n\right) \\
        &= T\left(a_1u_1+\cdots+a_mu_m\right)+T\left(b_1v_1+\cdots+b_nv_n\right) \\
        &= b_1Tv_1+\cdots+b_nTv_n
    \end{aligned}$$
    若$\dim W<n$,必然$Tv_1,\cdots,Tv_n$线性相关,于是存在这样的$b_1,\cdots,b_n$使得$Tv=\mbf{0}$,矛盾.\\
    于是$\dim W\geqslant n$,即$\dim U\geqslant\dim V-\dim W$.\\
    综上,命题得证.
\end{proof}
\begin{problem}[19.]
    设$W$是有限维的,$T\in\mathcal{L}(V,W)$.试证明:$T$是单射,当且仅当存在$S\in\mathcal{L}(W,V)$使得$ST$是$V$上的恒等算子.
\end{problem}
\begin{proof}
    若$T$是单射,对于任意$w\in\range T$都存在唯一的$v\in V$使得$Tv=w$.\\
    设$\range T$的基为$w_1,\cdots,w_m$,它可以被扩展为$W$的一组基$w_1,\cdots,w_m,u_1,\cdots,u_n$.\\
    定义映射$S:V\to W$满足
    $$\left\{\begin{array}{l}
        Sw_k=v_k,k\in\left\{1,\cdots,m\right\}\text{且}Tv_k=w_k\\
        Su_j=\mbf{0},j\in\left\{1,\cdots,n\right\}
    \end{array}\right.$$
    容易验证$S$是线性映射.\\
    于是对于任意$v\in V$都有$ST(v)=S(Tv)=Sw=v$,即存在这样的$S$使得$ST$为$V$上的恒等算子.\\
    若存在$S\in\mathcal{L}(W,V)$使得$ST$为$V$上的恒等算子,那么对于任意$v_1,v_2\in V$有
    $$STv_1=S(Tv_1)=v_1,STv_2=S(Tv_2)=v_2$$
    于是$Tv_1=Tv_2$当且仅当$v_1=v_2$(否则$S$会将一个向量映射为两个不同的值),从而$T$是单射.\\
    综上,命题得证.
\end{proof}
\begin{problem}[20.]
    设$W$是有限维的,$T\in\mathcal{L}(V,W)$.试证明:$T$是满射,当且仅当存在$S\in\mathcal{L}(W,V)$使得$TS$是$W$上的恒等算子.
\end{problem}
\begin{proof}
    若$T$是满射,于是对于任意$w\in W$都存在$v\in V$使得$Tv=w$.\\
    定义映射$S:W\to V$满足$\forall w\in W,Sw=v$其中$Tv=w$(如果存在多个满足条件的$v$,则任取一个即可).容易验证$S$是线性的.\\
    于是$TSw=Tv=w$,从而存在这样的$S$使得$TS$是$W$上的恒等算子.\\
    若存在线性映射$S$使得$TS$是$W$上的恒等算子,那么对于任意$w\in W$均有$TSw=T(Sw)=w$.\\
    于是$\range T=W$,因而$T$是满射.\\
    综上,命题得证.
\end{proof}
\begin{problem}[21.]
    设$V$是有限维的,$T\in\mathcal{L}(V,W)$,$U$是$W$的子空间.
    试证明:集合$\left\{v\in V:Tv\in U\right\}$是$V$的子空间,且满足
    $$\dim\left\{v\in V:Tv\in U\right\}=\dim\nul T+\dim(U\cap\range T)$$
\end{problem}
\begin{proof}
    我们先证明$\left\{v\in V:Tv\in U\right\}$是$V$的子空间.记这个集合为$S$.\\
    对于$v_1,v_2\in S$,不妨令$Tv_1=u_1,Tv_2=u_2$.因为$U$是$W$的子空间,于是
    $$T(v_1+v_2)=Tv_1+Tv_2=u_1+u_2\in U$$
    从而$v_1+v_2\in S$,即$S$对加法封闭.\\
    验证$S$对乘法封闭也是类似的,在这里略去.\\
    未完待续.
\end{proof}
\begin{problem}[22.]
    设$U$和$V$是有限维向量空间,$S\in\mathcal{L}(V,W)$和$T\in\mathcal{L}(U,V)$,试证明
    $$\dim\nul ST\leqslant\dim\nul S+\dim\nul T$$
\end{problem}
\begin{proof}
    根据线性映射基本定理,我们有
    $$\begin{aligned}
        \dim\nul ST
        &= \dim U-\dim\range ST \\
        &= \dim\nul T+\dim\range T-\dim\range ST \\
        &\leqslant \dim\nul T+\dim V-\dim\range ST \\
        &= \dim\nul T+\dim\nul S+\dim\range S-\dim\range ST
    \end{aligned}$$
    我们只需证明$\dim\range S\geqslant\dim\range ST$.\\
    对于任意$w\in\range ST$,都存在$u\in U$使得$STu=w$.于是对于任意$w$总存在
    $V$中的向量$Tu$使得$S$将其映射到$w$.从而$\range ST\subseteq\range S$,于是$\dim\range ST\leqslant\dim\range S$,命题得证.
\end{proof}
\begin{problem}[23.]
    设$U$和$V$是有限维向量空间,$S\in\mathcal{L}(V,W)$和$T\in\mathcal{L}(U,V)$,试证明
    $$\dim\range ST\leqslant\min\left\{\dim\range S,\dim\range T\right\}$$
\end{problem}
\begin{proof}
    在\tbf{22.}中我们已经证明了$\dim\range ST\leqslant\dim\range S$.\\
    设$\li v,m$是$\range T$的一组基.容易知道$\range ST=\text{span}\left(\li {Tv},m\right)$.\\
    于是我们有$\dim\range ST=\dim\range\text{span}\left(\li{Tv},m\right)\leqslant m$.\\
    这就证明了$\dim\range ST\leqslant\dim\range T$.\\
    命题得证.
\end{proof}
\begin{problem}[24.]
    回答下列问题.
    \begin{enumerate}[label=\tbf{(\arabic*)}]
        \item 设$\dim V=5$,且$S,T\in\mathcal{L}(V)$使得$ST=\mbf{0}$.试证明:$\dim\range TS\leqslant 2$.
        \item 给出一例:$S,T\in\mathcal{L}(\F^5),ST=\mbf{0}$且$\dim\range ST=2$.
    \end{enumerate}
\end{problem}
\begin{solution}
    \begin{enumerate}[label=\tbf{(\arabic*)}]
        \item \textbf{Proof}.\\
            假定$\dim\range TS>2$.
            根据\tbf{23.}可得$\dim\range T>2,\dim\range S>2$.\\
            从而$\dim\nul T\leqslant 2,\dim\nul S\leqslant 2$.\\
            由于$\dim\nul T\leqslant 2$,于是至多有两个基向量$v_k$满足$Tv_k=\mbf{0}$.\\
            设$\li v,5$为$V$的一组基.设$v=\li v+5\in V$.\\
            不妨假定$\li {Tv},3\neq\mbf{0}$,于是$STv=\li {STv}+3$.\\
            由于$\li v,5$线性无关,于是$\li {Tv},3$线性无关.\\
            由于$\dim\nul S\leqslant 2$,于是至多有两个$Tv_k$满足$S(Tv_k)=\mbf{0}$.\\
            不妨假定$STv_1\neq\mbf{0}$,这与$ST=\mbf{0}$矛盾.于是命题得证.
        \item \tbf{Solution.}\\
            略.
    \end{enumerate}
\end{solution}
\begin{problem}[25.]
    设$W$是有限维的,$S,T\in\mathcal{L}(V,W)$.证明:$\nul S\subseteq\nul T$,当且仅当存在$E\in\mathcal{L}(W)$使得$T=ES$.
\end{problem}
\begin{proof}
    当$\nul S\subseteq\nul T$时,假定$\li v,m$是$\nul S$的一组基,$\li v,m,\li u,n$是$\nul T$的一组基.\\
    再假定$\li v,m,\li u,n,\li w,k$是$V$的一组基.
    
\end{proof}
\end{document}