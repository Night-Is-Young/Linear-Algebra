\documentclass{ctexart}
\usepackage{geometry}
\usepackage[dvipsnames,svgnames]{xcolor}
\usepackage[strict]{changepage}
\usepackage{framed}
\usepackage{enumerate}
\usepackage{amsmath,amsthm,amssymb}
\usepackage{enumitem}
\usepackage{template}

\allowdisplaybreaks
\geometry{left=2cm, right=2cm, top=2.5cm, bottom=2.5cm}

\begin{document}
\pagestyle{empty}
\begin{center}\large 张成空间\end{center}
\tbf{1.线性组合与张成空间}\\
在本章中将研究向量组(即由向量构成的组).为区别于数构成的组,我们在表示时不再用圆括号括起来.\\
例如,$(1,2),(0,3)$表示由向量$(1,2)$和$(0,3)$构成的组.\\
将一个向量组中的向量分别进行标量乘法后求和的结果就称为该向量组的线性组合.
\begin{definition}[1.1 定义:线性组合]
    $V$中一个向量组$v_1,\cdots,v_m$的\tbf{线性组合}是形如
    $$a_1v_1+\cdots+a_mv_m$$的向量,其中$a_1,\cdots,a_m\in\F$.
\end{definition}\noindent
任意改变上述定义中的$a_1,\cdots,a_m$的值,可以得到一类向量.这就引出了\tbf{张成空间}的定义.
\begin{definition}[1.2 定义:张成空间]
    $V$中一个向量组$v_1,\cdots,v_m$的所有线性组合构成的集合称为$v_1,\cdots,v_m$的\tbf{张成空间},
    记作$\text{span}(v_1,\cdots,v_m)$,即
    $$\text{span}(v_1,\cdots,v_m)=\left\{a_1v_1+\cdots+a_mv_m:a_1,\cdots,a_m\in\F\right\}$$
    额外定义空向量组$()$的张成空间为$\left\{\mbf{0}\right\}$.
\end{definition}\noindent
关于张成空间有如下定理.
\begin{formal}[1.3 张成空间的性质]
    $V$中向量组的张成空间是最小的包含这向量组中所有向量的$V$的子空间.
\end{formal}
\begin{solution}[Proof.]
    假定$v_1,\cdots,v_m$为$V$中的向量组.\\
    首先我们说明$\text{span}(v_1,\cdots,v_m)$是$V$的子空间.
    \begin{enumerate}[label=\tbf{(\arabic*)}]
        \item 加法恒等元$\mbf{0}$属于$\text{span}(v_1,\cdots,v_m)$,因为$$\mbf{0}=0v_1+\cdots+0v_m$$
        \item $\text{span}(v_1,\cdots,v_m)$对加法封闭,因为$$(a_1v_1+\cdots+a_mv_m)+(b_1v_1\cdots+b_mv_m)=(a_1+b_1)v_1+\cdots+(a_m+b_m)v_m$$
        \item $\text{span}(v_1,\cdots,v_m)$对标量乘法封闭,因为$$\lambda(a_1v_1+\cdots+a_mv_m)=(\lambda a_1)v_1+\cdots+(\lambda a_m)v_m$$
    \end{enumerate}
    于是$\text{span}(v_1,\cdots,v_m)$为$V$的子空间.\\
    对于任意$v_k(1\leqslant k\leqslant m)$有$v_k\in \text{span}(v_1,\cdots,v_m)$.
    反之,由于每个子空间对标量乘法和加法封闭,于是包含所有$v_k$的子空间一定包含$\text{span}(v_1,\cdots,v_m)$,
    从而$\text{span}(v_1,\cdots,v_m)$是最小的包含所有$v_k$的子空间.\\
    综上,原命题得证.
\end{solution}\noindent
\tbf{2.张成与向量空间的维度}\\
在张成空间中有一类特殊的结果.
\begin{definition}[2.1 定义:张成]
    若$\text{span}(v_1,\cdots,v_m)=V$,我们称$v_1,\cdots,v_m$\tbf{张成}$V$.
\end{definition}\noindent
据此,我们给出一个关键的定义.
\begin{definition}[2.2 定义:有限维向量空间]
    如果向量空间$V$可由$V$中某个向量组张成,则称该向量空间$V$是有限维的.
\end{definition}\noindent
不难发现对于每个$n\in\N^*$,$\F^n$都是有限维的.
为了给出一个无限维向量空间的例子,我们回顾一下多项式的定义.
\begin{definition}[2.3 定义:多项式,$\mathcal{P}(\F)$]
    对一个函数$p:\F\to\F$,如果存在$a_0,\cdots,a_m\in\F$使得
    $$\forall z\in\F,p(z)=a_0+a_1z+a_2z^2+\cdots+a_mz^m$$
    则称$p$为系数在$\F$中的多项式.\\
    对应的,称$\mathcal{P}(\F)$为系数在$\F$中的全体多项式所构成的集合.
\end{definition}\noindent
带有通常的加法和标量乘法的$\mathcal{P}(\F)$是一个向量空间.特别地,它还是$\F^\F$的子空间.\\
容易证明,一个多项式的系数由该多项式唯一决定.即:不存在用两组不同的系数表示同一个多项式的方法.
于是下面的定义唯一地规定了一个多项式的次数.
\begin{definition}[2.4 定义:多项式的次数,$\text{deg }p$]
    对于一个多项式$p\in\mathcal{P}(\F)$,如果存在$a_0,a_1,\cdots,a_m\in\F$且$a_m\neq0$使得
    $$\forall z\in\F,p(z)=a_0+a_1z+a_2z^2+\cdots+a_mz^m$$
    则称$p$的\tbf{次数}为$m$.\\
    规定恒等于$0$的多项式的次数为$-\infty$.\\
    相应地,记多项式$p$的次数为$\text{deg }p$.
\end{definition}
\begin{definition}[2.5 记号:$\mathcal{P}_m(\F)$]
    对于非负整数$m$,$\mathcal{P}_m(\F)$表示$\mathcal{P}(\F)$中所有次数不大于$m$的多项式构成的集合.
\end{definition}\noindent
在上面的定义中,我们约定$-\infty<m$,于是多项式$0$包含于任意$\mathcal{P}_m(\F)$中.\\
记函数$f^k(z)=z^k$,不难说明$\mathcal{P}_m(\F)=\text{span}(f^0,f^1,\cdots,f^m)$.
于是,对于每个非负整数$m$,$\mathcal{P}_m(\F)$都是有限维向量空间.相应的,我们给出无限维向量空间的定义.
\begin{definition}[2.6 定义:无限维向量空间]
    如果一个向量空间不是有限维的,那么就称它为\tbf{无限维的}.
\end{definition}\noindent
\begin{problem}[2.7 例]
    证明:$\mathcal{P}(\F)$是无限维向量空间.
\end{problem}
\begin{solution}[Proof.]
    考虑$\mathcal{P}(\F)$中的任意一组元素,设其中次数最高的多项式的次数为$m$,
    从而任意次数高于$m$的多项式均不能用该组元素线性表出.
    于是,没有组能张成$\mathcal{P}(\F)$,故$\mathcal{P}(\F)$是无限维向量空间.
\end{solution}
\end{document}