\documentclass{ctexart}
\usepackage{geometry}
\usepackage[dvipsnames,svgnames]{xcolor}
\usepackage[strict]{changepage}
\usepackage{framed}
\usepackage{enumerate}
\usepackage{amsmath,amsthm,amssymb}
\usepackage{enumitem}
\usepackage{template}
\usepackage{nicematrix}

\allowdisplaybreaks
\linespread{1.5}
\geometry{left=2cm, right=2cm, top=2.5cm, bottom=2.5cm}

\begin{document}
\pagestyle{empty}
\begin{center}\large 矩阵乘法\end{center}
\tbf{1.矩阵乘法}\\
我们知道两个线性映射$S,T$的乘积(实际上就是函数的复合)记作$ST$,也有相关的结合律和分配律(尽管并不满足交换律).
那么它们的矩阵之间也有相应的乘法吗?下面我们来定义矩阵之间的乘法.
假定$v_1,\cdots,v_n$是$V$的基,$w_1,\cdots,w_m$是$W$的基,$u_1,\cdots,u_p$是$U$的基.
考虑线性映射$T:U\to V$和$S:V\to W$,复合映射$ST$是一个$U$到$W$的映射.我们来考虑如何表示$ST$.\\
假定$\mathcal{M}(S)=A$且$\mathcal{M}(T)=B$.对于$1\leqslant k\leqslant p$,我们有
$$\begin{aligned}
    (ST)u_k
    &= S\left(\sum_{r=1}^{n}B_{r,k}v_r\right) \\
    &= \sum_{r=1}^{n}B_{r,k}Sv_r \\
    &= \sum_{r=1}^{n}B_{r,k}\left(\sum_{j=1}^{m}A_{j,r}w_j\right) \\
    &= \sum_{j=1}^{m}\left(\sum_{r=1}^{n}A_{j,r}B{r,k}\right)w_j
\end{aligned}$$
这样我们便希望$\mathcal{M}(ST)$是一个$m\times p$矩阵,使得第$j$行第$k$列的元素为
$$\sum_{r=1}^{n}A_{j,r}B_{r,k}$$
现在我们就来定义满足我们所期望的含义的矩阵乘法.
\begin{definition}[1.1 矩阵乘法]
    假设$A$是$m\times n$矩阵,$B$是$n\times p$矩阵,那么$A$和$B$的\tbf{乘积}$AB$定义为一个$m\times p$矩阵,其中第$j$行第$k$列的元素由
    $$(AB)_{j,k}=\sum_{r=1}^{n}A_{j,r}B_{r,k}$$
    给出.于是取$A$的第$j$行和$B$的第$k$列,将它们对应位置上的元素相乘后相加即得到$AB$的第$j$行第$k$列的元素.
\end{definition}\noindent
这样我们就可以知道有如下定理.
\begin{formal}[1.2 线性映射之积的矩阵]
    如果$T\in\mathcal{L}(U,V)$且$S\in\mathcal{L}(V,W)$,那么$\mathcal{M}(ST)=\mathcal{M}(S)\mathcal{M}(T)$.
\end{formal}\noindent
上面结果的证明就是我们在定义矩阵乘法之前描述我们的动机时所做的运算.\\
我们还需要引入如下的记号.
\begin{definition}[1.3 记号:$A_{j,\cdot}$与$A_{\cdot,k}$]
    假设$A$是一个$m\times n$矩阵.
    \begin{enumerate}[label=\tbf{(\alph*)}]
        \item 如果$1\leqslant j\leqslant m$,那么$A_{j,\cdot}$表示由$A$的第$j$行构成的$1\times n$矩阵.
        \item 如果$1\leqslant k\leqslant n$,那么$A_{\cdot,k}$表示由$A$的第$k$列构成的$m\times 1$矩阵.
    \end{enumerate}
\end{definition}\noindent
$1\times n$矩阵和$n\times 1$矩阵的乘积是$1\times 1$矩阵,然而我们也通常把$1\times 1$矩阵与其元素等同起来看.例如
$$\begin{pmatrix}3 & 4\end{pmatrix}\begin{pmatrix}6\\2\end{pmatrix}=(26)\text{ 或 }\begin{pmatrix}3 & 4\end{pmatrix}\begin{pmatrix}6\\2\end{pmatrix}=26$$
接下来的结论就采用了上面的讨论中的约定,从而给出了另一种思考矩阵乘法的方式.
\begin{formal}[1.4 矩阵之积的元素等于行乘以列]
    假设$A$是$m\times n$矩阵,$B$是$n\times p$矩阵.那么若$1\leqslant j\leqslant m$且$1\leqslant k\leqslant p$,则
    $$(AB)_{j,k}=A_{j,\cdot}B_{\cdot,k}$$
    换言之,$AB$的第$j$行第$k$列的元素等于$A$的第$j$行乘以$B$的第$k$列.
\end{formal}\noindent
接下来的结论又给出了另一种思考矩阵乘法的方式.
\begin{formal}[1.5 矩阵之积的列等于矩阵与列之积]
    假设$A$是$m\times n$矩阵,$B$是$n\times p$矩阵.那么若$1\leqslant k\leqslant p$,则
    $(AB)_{\cdot,k}=AB_{\cdot,k}$
    换言之,$AB$的第$k$列等于$A$乘以$B$的第$k$列.
\end{formal}\noindent
上面的式子告诉我们,矩阵的乘积可以看作列的线性组合.
\begin{formal}[1.6 列的线性组合]
    假设$A$是$m\times n$矩阵,$b=\begin{pmatrix}b_1\\\vdots\\b_n\end{pmatrix}$是$n\times 1$矩阵,那么
    $$Ab=b_1A_{\cdot,1}+\cdots+b_nA_{\cdot,n}$$
    换言之,$Ab$是$A$中各列的线性组合,而与这些列相乘的标量来自$b$.
\end{formal}\noindent
于是根据上面的思想,我们可以把矩阵乘法视作列或行的线性组合.
\begin{formal}[1.7 将矩阵乘法视作列或行的线性组合]
    假设$C$是$m\times c$矩阵,$R$是$c\times n$矩阵.
    \begin{enumerate}[label=\tbf{(\alph*)}]
        \item 对于$k\in\left\{1,\cdots,n\right\}$,$CR$的第$k$列是$C$的各列的线性组合,各系数来自$R$的第$k$列.
        \item 对于$j\in\left\{1,\cdots,m\right\}$,$CR$的第$j$行是$R$的各行的线性组合,各系数来自$C$的第$j$行.
    \end{enumerate}
\end{formal}
\end{document}