\documentclass{ctexart}
\usepackage{geometry}
\usepackage[dvipsnames,svgnames]{xcolor}
\usepackage[strict]{changepage}
\usepackage{framed}
\usepackage{enumerate}
\usepackage{amsmath,amsthm,amssymb}
\usepackage{enumitem}
\usepackage{template}
\usepackage{nicematrix}

\allowdisplaybreaks
\linespread{1.5}
\geometry{left=2cm, right=2cm, top=2.5cm, bottom=2.5cm}

\begin{document}
\pagestyle{empty}
\begin{center}\large 可逆性和同构\end{center}
\tbf{1.可逆线性映射}\\
我们先从线性映射的可逆和逆的概念开始.
\begin{definition}[1.1 定义:可逆的,逆]
    对于线性映射$T\in\mathcal{L}(V,W)$,如果存在线性映射$S\in\mathcal{L}(W,V)$使得$ST$等于$V$上的恒等算子且$TS$等于$W$上的恒等算子,
    则称$T$是\tbf{可逆的},并称这样的$S$是$T$的一个\tbf{逆}.
\end{definition}\noindent
我们在上面中说的是"一个逆".然而"逆"的数量是否有限制呢?下面的定理告诉我们线性映射的逆是唯一的.
\begin{formal}[1.2 逆的唯一性]
    可逆的线性映射具有唯一的逆.
\end{formal}
\begin{proof}
    假设$T\in\mathcal{L}(V,W)$是可逆的,且$S_1$和$S_2$均为$T$的逆,那么
    $$S_1=S_1I=S_1(TS_2)=(S_1T)S_2=IS_2=S_2$$
    这表明$S_1=S_2$,于是线性映射的逆是唯一的.
\end{proof}\noindent
既然如此,我们可以给逆一个记号.
\begin{definition}[1.3 记号:逆]
    如果$T$是可逆的,那么记它的逆为$T^{-1}$.
\end{definition}\noindent
我们在单射和满射一节中的习题中已经看到了类似于逆的概念.以下的定理表明了线性映射可逆所需的条件.
\begin{formal}[1.4 可逆的判定]
    一个线性映射可逆,当且仅当它既是单射又是满射.
\end{formal}
\begin{proof}
    假设$T\in\mathcal{L}(V,W)$.我们需要证明$T$可逆当且仅当$T$既是单射又是满射.\\
    首先假设$T$是可逆的,现在先证明$T$是单射.假定$u,v\in V$且$Tu=Tv$,于是
    $$u=Iu=T^{-1}(Tu)=T^{-1}(Tv)=Iv=v$$
    从而$u=v$,于是$T$是单射.\\
    仍然假设$T$是可逆的,现在证明$T$是满射.对于任意$w\in W$都有
    $$w=Iw=T(T^{-1}w)$$
    这就表明$w\in\range T$,于是$\range T=W$,从而$T$是满射.\\
    现在假设$T$既是单射又是满射,证明$T$是可逆的.\\
    对于任意$w\in W$,定义映射$S$使得$S(w)\in V$是$V$中唯一使得$T(S(w))=w$成立的元素
    (这样一个元素的存在性来源于$T$的满射性,唯一性来源于$T$的单射性).于是$T\circ S$是$W$上的恒等算子.\\
    我们先来证明$S\circ T$是$V$上的恒等算子.对于任意$v\in V$有
    $$T(S\circ T(v))=T\circ S(Tv)=Tv$$
    由于$T$是单射,于是$S\circ T(v)=v$,于是$S\circ T$是$V$上的恒等算子.\\
    现在只需证明这样的$S$是线性映射即可.\\
    对于任意$w_1,w_2\in W$,我们有
    $$T(S(w_1)+S(w_2))=T(S(w_1))+T(S(w_2))=w_1+w_2$$
    这表明$S(w_1)+S(w_2)$被$T$唯一映射到$w_1+w_2$.于是$S(w_1+w_2)=S(w_1)+S_(w_2)$,从而$S$具有可加性.\\
    对于任意$\lambda\in\F$和任意$w\in W$,我们有
    $$T(\lambda S(w))=\lambda T(S(w))=\lambda w$$
    这表明$\lambda S(w)$被$T$唯一映射到$\lambda w$.于是$S(\lambda w)=\lambda S(w)$,从而$S$具有齐次性.\\
    综合上面的所有证明可知原命题成立.
\end{proof}\noindent
你可能会好奇,对于同维线性映射之间的映射,其可逆性是否一定需要同时满足上述两个条件来满足呢?
这有些类似于集合中的等势概念.我们首先需要说明,对于无限维的向量空间之间的映射,仅仅满足一条是不够的.\\
令$S\in\mathcal{L}(\mathcal{P}(\R),\mathcal{P}(\R))$满足$\forall p\in\mathcal{P}(\R),Sp=x^2p$.这样的$S$是单射,但是它并不是满射,因为多项式$1\notin\range S$.\\
令$S\in\mathcal{L}(\F^{\infty},\F^{\infty})$是后向位移映射,这样的$S$是满射,但不是单射,因为$S(1,0,\cdots)=S(0,0,\cdots)=\mbf{0}$.\\
上述两例中的$S$因而均不是可逆的.然而对于有限维空间,情形则就不同了.
\begin{formal}[1.5 单射,满射与可逆的等价性]
    若$\dim V=\dim W$有限,且$T\in\mathcal{L}(V,W)$,
    那么\tbf{(a)}$T$是可逆的\tbf{(b)}$T$是单射\tbf{(c)}$T$是满射\ 三者是等价的.
\end{formal}
\begin{proof}
    由线性映射基本定理$$\dim\range T+\dim\nul=\dim V$$
    若$T$是单射,可知$\dim\nul T=0$.从而
    $$\dim\range T=\dim V-\dim\nul T=\dim V=\dim W$$
    于是$\range T=W$,故$T$是满射.\\
    若$T$是满射,可知$\dim\range T=\dim W$,从而
    $$\dim\nul T=\dim V-\dim\range T=\dim V-\dim W=0$$
    于是$T$是单射.\\
    从$T$是单射或$T$是满射可以推出另一性质,进而可以推出$T$是可逆的.\\
    综上,我们证明了三者是等价的.
\end{proof}\noindent
运用线性代数,我们可以简洁地证明一些看起来和线性代数毫无关系的命题.例如:\\
试证明:对于任意多项式$q$,都存在多项式$p$使得$\dfrac{\di^2\left[\left(x^2+5x+7\right)p\right]}{\dx^2}=q$.
\begin{proof}
    设$q$的次数为$m$.从$\mathcal{P}_m{\F}$到$\mathcal{P}_m(\F)$的线性映射$p\mapsto \left[\left(x^2+5x+7\right)p\right]''$是单射.\\
    因而它是满射,于是一定存在$p\in\F^m$使得题设式子成立.
\end{proof}\noindent
我们另有一个满足乘法交换律的线性映射的例子.
\begin{formal}[1.6 满足乘法交换律的线性映射]
    假设$V$和$W$是有限维向量空间且$\dim V=\dim W$,$S\in\mathcal{L}(W,V)$且$T\in\mathcal{L}(V,W)$.\\
    那么$ST=I$当且仅当$TS=I$.
\end{formal}
\begin{proof}
    先设$ST=I$.如果$v\in V$且$Tv=\mbf{0}$,那么
    $$v=Iv=(ST)v=S(Tv)=S(\mbf{0})=\mbf{0}$$
    于是$T$是单射.由\tbf{1.5}可得$T$是可逆的.\\
    对于$ST$两端同时乘以$T^{-1}$有$S=T^{-1}$,于是$TS=TT^{-1}=I$.\\
    交换上面证明中的$S,T$和$V,W$即可证明另一方向的逻辑关系.\\
    综上,命题得证.
\end{proof}\noindent
\tbf{2.同构向量空间}\\
接下来的定义有助于描述两个除了元素名称不同外本质上相同的向量空间.
\begin{definition}[2.1 定义:同构,同构的]
    一个\tbf{同构}就是一个可逆线性映射.对于两个向量空间,若存在将其中一个向量空间映射成另一个向量空间的同构,则称这两个向量空间是\tbf{同构的}.
\end{definition}\noindent
我们说在本质上相同,是因为对于同构$T:V\to W$,对于任意$v\in V$,我们都可以把它改写成$Tv\in W$,且不会改变其运算性质.\\
对于两个数学结构,判断它们在本质上相同有时候是很困难的,然而对于向量空间则是简单的.具体来说,我们观察两个向量空间的维数即可.
\begin{formal}[2.2 维数表明向量空间是否同构]
    对于$\F$上的两个有限维向量空间$V$和$W$,当且仅当$\dim V=\dim W$时,$V$和$W$同构.
\end{formal}
\begin{proof}
    假定$V$和$W$同构,那么存在同构$T:V\to W$.由于$T$是可逆的,于是$\nul T=\left\{\mbf{0}\right\}$且$\range T=W$.\\
    于是$\dim V=\dim\nul T+\dim\range T=0+\dim W=\dim W$.\\
    假定$\dim V=\dim W$,于是取$V$的一个基$v_1,\cdots,v_m$和$W$的一个基$w_1,\cdots,w_m$.定义$T\in\mathcal{L}(V,W)$为
    $$T\left(a_1v_1+\cdots+a_mv_m\right)=a_1w_1+\cdots+a_mw_m$$
    由于$w_1,\cdots,w_m$张成$W$,于是$T$是满射;由于$w_1,\cdots,w_m$线性无关,于是$T$是单射.\\
    于是$T$是同构,进而$V$和$W$同构.
\end{proof}\noindent
这个结论表明,对于任意一个有限维向量空间,我们总能找到$\F^n$与其同构.\\
注意到对于两个有限维向量空间$V$和$W$且$\dim V=n,\dim W=m$,在选定基的情况下,
对于任意$T\in\mathcal{L}(V,W)$都可以将其写作矩阵$\mathcal{M}(T)\in\F^{m,n}$的形式.
更一般的来说,$\mathcal{M}$将向量空间$\mathcal{L}(V,W)$映射到了向量空间$\F^{m,n}$上.\\
我们似乎可以发现这两个向量空间之间的关系.
\begin{formal}[2.3 $\mathcal{L}(V,W)$与$\F^{m,n}$同构]
    设$v_1,\cdots,v_n$是$V$的基且$w_1,\cdots,w_m$是$W$的基,
    那么$\mathcal{M}$是$\mathcal{L}(V,W)$与$\F^{m,n}$间的同构.
\end{formal}
\begin{proof}
    我们已经注意到$\mathcal{M}$是线性的.现在只需证明$\mathcal{M}$既是单射又是满射.\\
    若$T\in\mathcal{L}(V,W)$且$\mathcal{M}(T)=\mbf{0}$,那么对于任意$k\in\left\{1,\cdots,n\right\}$均有$Tv_k=\mbf{0}$.
    因为$v_1,\cdots,v_n$是$V$的基,所以$T=\mbf{0}$,进而$\nul\mathcal{M}=\left\{\mbf{0}\right\}$,于是$\mathcal{M}$是单射.\\
    对于任意$A\in\F^{m,n}$,存在$T\in\mathcal{L}(V,W)$使得
    $$\forall k\in\left\{1,\cdots,n\right\},Tv_k=\sum_{j=1}^{m}A_{j,k}w_j$$
    于是$\mathcal{M}(T)=A$,故$\range\mathcal{M}=\F^{m,n}$,进而$\mathcal{M}$是满射.\\
    综上,$\mathcal{M}$是同构,命题得证.
\end{proof}\noindent
现在我们可以确定线性映射所构成的向量空间的维数了.
\begin{formal}[2.4 线性映射的向量空间的维数]
    假设$V$和$W$是有限维向量空间,那么$\mathcal{L}(V,W)$是有限维的,满足
    $$\dim\left(\mathcal{L}(V,W)\right)=\left(\dim V\right)\left(\dim W\right)$$
\end{formal}\noindent
证明是简单的,这里就略去.
\end{document}