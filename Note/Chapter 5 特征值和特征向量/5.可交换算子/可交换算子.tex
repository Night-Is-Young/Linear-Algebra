\documentclass{ctexart}
\usepackage{geometry}
\usepackage[dvipsnames,svgnames]{xcolor}
\usepackage[strict]{changepage}
\usepackage{framed}
\usepackage{enumerate}
\usepackage{amsmath,amsthm,amssymb}
\usepackage{enumitem}
\usepackage{template}

\allowdisplaybreaks
\geometry{left=2cm, right=2cm, top=2.5cm, bottom=2.5cm}

\begin{document}
\pagestyle{empty}
\begin{center}\large 可交换算子\end{center}
\tbf{1.可交换}
\begin{definition}[1.1 定义:可交换]
    \begin{enumerate}[label=\tbf{(\arabic*)}]
        \item 对于同一向量空间上的两个算子$S$和$T$,若$ST=TS$,则称它们\tbf{可交换}.
        \item 对于两个大小相同的方阵,若$AB=BA$,则称它们\tbf{可交换}.
    \end{enumerate}
\end{definition}\noindent
自然地,我们有如下定理.
\begin{formal}[1.2 可交换算子对应可交换矩阵]
    设$S,T\in\L(V)$且$\li v,n$是$V$的一个基,那么$S$和$T$可交换,当且仅当$\mathcal{M}(S)$和$\mathcal{M}(T)$可交换.
\end{formal}
\begin{formal}[1.3 特征空间在可交换算子下不变]
    设$S,T\in\L(V)$可交换且$\lambda\in F$,那么$E(\lambda,S)$在$T$下不变.
\end{formal}
\begin{proof}
    设$v\in E(\lambda,S)$,那么
    $$S(Tv)=(ST)v=(TS)v=T(Sv)=T(\lambda v)=\lambda Tv$$
    这表明$Tv\in E(\lambda,S)$,于是$E(\lambda,S)$在$T$下不变.
\end{proof}
\begin{formal}[1.4 可同时对角化等价于可交换]
    同一向量空间上的两个可对角化算子关于相同的基有对角矩阵当且仅当这两个算子可交换.
\end{formal}
\begin{proof}
    $\Rightarrow$:根据矩阵乘法的定义,两个对角矩阵的乘积也是对角矩阵,其对角线元素等于两个矩阵对应位置的对角线元素的乘积.于是这两个矩阵可交换,因而这两个算子可交换.\\
    $\Leftarrow$:设$S,T\in\L(V)$是可对角化算子且可交换.令$\li\lambda,m$表示$S$的互异特征值.因为$S$可对角化,于是
    $$V=E(\lambda_1,S)\oplus\cdots\oplus E(\lambda_m,S)$$
    又$E(\lambda_k,S)$在$T$下不变,于是对于每个$T|_{E(\lambda_k,S)}$均可对角化.%
    于是对于每个$k$都存在$T$的特征向量组成的$E(\lambda_k,S)$的基.%
    将这些基合并,即可得到$V$的一个基,且该基中的每个向量都同时是$S$和$T$的特征向量.于是$S$和$T$关于这组基有对角矩阵,命题得证.
\end{proof}
\begin{formal}[1.5 可交换算子的公共特征向量]
    非零有限维复空间向量上的每对可交换算子都具有公共的特征向量.
\end{formal}
\begin{proof}
    从\tbf{1.4}的证明即可看出.
\end{proof}
\begin{formal}[1.6 可交换算子可同时上三角化]
    设$V$是有限维复向量空间,$S,T$是$V$上的可交换算子,那么存在$V$的一个基使得$S,T$关于该基有上三角矩阵.
\end{formal}
\begin{proof}
    令$n=\dim V$.我们对$n$用归纳法.\\
    欲证结论对$n=1$是成立的,因为所有$1\times 1$矩阵都是上三角矩阵.\\
    现在假设$n>1$,且欲证结论对所有维数小于$n$的情况都成立.\\
    设$v_1$为$S,T$共有的特征向量,于是$Sv_1,Tv_1\in\span(v_1)$.令$W$是$V$的子空间使得$V=W\oplus\span(v_1)$.\\
    定义$P\in\L(V,W)$为:对任意$a\in C$和任意$w\in W$有$P(av_1+w)=w$.\\
    定义$\tilde{S},\tilde{T}\in\L(W)$为:对任意$w\in W$有$\tilde{S}w=P(Sw),\tilde{T}w=P(Tw)$.\\
    对于任意$w\in W$,存在$a\in C$使得
    $$(\tilde{S}\tilde{T})w=\tilde{S}(P(Tw))=\tilde{S}(Tw-av_1)=P(S(Tw-av_1))=P((ST)w)$$
    同理可知$(\tilde{T}\tilde{S})w=P((TS)w)$.因为$S,T$可交换,于是$P((ST)w)=P((TS)w)$.\\
    于是$(\tilde{S}\tilde{T})w=(\tilde{T}\tilde{S})w$,因此$(\tilde{S}$和$\tilde{T})w$可交换.\\
    于是根据归纳假设,存在$W$的一个基$v_2,\cdots,v_n$使得$(\tilde{S}$和$\tilde{T})w$关于该基有上三角矩阵.此时$\li v,n$就是$V$的基.\\
    对于任意$2\leqslant k\leqslant n$,存在$a_k,b_k\in C$使得
    $$Sv_k=a_kv_1+\tilde{S}v_k\text{  且  }Tv_k=a_kv_1+\tilde{T}v_k$$
    因为$(\tilde{S}$和$\tilde{T})w$关于$v_2,\cdots,v_k$有上三角矩阵,于是$\tilde{S}v_k,\tilde{T}v_k\in\span(v_2,\cdots,v_k)$.\\
    结合上式可知$Sv_k,Tv_k\in\span(\li v,k)$.这表明$S,T$关于$\li v,n$有上三角矩阵,命题得证.
\end{proof}
\begin{formal}[1.7 可交换算子的积与和的特征值]
    设$V$是有限维复向量空间,$S,T$是$V$上的可交换算子.那么
    \begin{enumerate}[label=\tbf{(\arabic*)}]
        \item $S+T$的每个特征值都等于$S$的某个特征值和$T$的某个之和.
        \item $ST$的每个特征值都等于$S$的某个特征值和$T$的某个之积.
    \end{enumerate}
\end{formal}\noindent
上面这一结论可以根据\tbf{1.6}和上三角矩阵的性质容易地得到.
\end{document}